\documentclass[12pt]{article}
\usepackage{amsmath}
\usepackage{amssymb}
\usepackage{graphicx}
\usepackage{color}

\begin{document}
\pagestyle{plain}

\title{}
\author{John DeSalvo}
\maketitle

\tableofcontents

\section{Abstract}

\section{Non-Technical Summary}

\section{Introduction}

\section{Background}

\section{Methods}
\subsection{Generating a Databse}
\sloppy
\definecolor{lightgray}{gray}{0.5}
\setlength{\parindent}{0pt}
\begin{verbatim}
function AC = ACDByear(year)
AC = [];
	for i = 1:12
        switch i
            case 1
                J = 31;
            case 2
                if mod(year, 4) == 0
                    J = 29;
                else
                    J = 28;
                end
            otherwise
               if sqrt(i) == round(sqrt(i)) || mod(i, 5) == 1
                   J = 30;
               else
                   J = 31;
               end
        end
        for j = 1:J
            ac = ACDBday(i, j, year);
            AC = vertcat(AC, ac);
        end
    end
end
\end{verbatim}

\begin{par}
This generates the database of information for all scheduled aircrafts on a given day. The number of customers in a day is random from a range, clustered around the max of that range close to Thanksgiving and Christmas. The objective is to minimize the amount of time each passenger spends waiting and maximize the amount of passengers in an AC. Suppose a flight is scheduled every 15 minutes.
\end{par} \vspace{1em}
\begin{verbatim}
function AC = ACDBday(month, day, year)
    actype = 100:50:400;
    % number of discrete time intervals, each is 15 minutes
    N = 96;
    f = @(x) abs(3*sin(3*x)./x) + pi/2;
    % determine the date of thanksgivng that year
    incr =  abs(year-2019);
    for i = 1:incr
        if mod(year, 4) == 0
            incr = incr + 2;
        else
            incr = incr + 1;
        end
    end
    % week of Thanksgiving
    if month == 11
        hday = 7 - mod(7, incr) + 21;
    % week of Christmas
    elseif month == 12 && abs(25-day) < 6
        hday = 25;
    else
        hday = 0;
    end

    for i = 1:N
        cap = randi([1 7]);
        n = randi([actype(cap)-50 actype(cap)+50]);
        for j = 1:5
            switch j
                case 1
                    AC(i, j) = actype(cap);
                case 2 % number of passengers
                    if hday
                        AC(i, j) = round(f(abs(day-hday)+1)*n/(exp(1)/2));
                    else
                        AC(i, j) = n;
                    end
                case 3 % number of passengers leaving at the next stop
                    AC(i,j) = randi([0 AC(i, j-1)]);
                case 4 % number of passengers boarding at the next stop
                    AC(i, j) = randi([0 AC(i, j-2) - AC(i, j-1)]);
                otherwise % timesetp, t
                    AC(i, j) = i;
            end
        end
    end
end
\end{verbatim}

\subsection{}

\section{Results}

\section{Conclusion}

\section{References}

\end{document}
